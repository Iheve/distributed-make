\documentclass[a4paper,11pt]{article}
%\pagestyle{headings}

% des paquetages indispensables, qui ajoutent des fonctionnalites
%\usepackage[latin1]{inputenc}
\usepackage[utf8]{inputenc}
%\usepackage[french]{babel}
%\usepackage{lmodern}
\usepackage{amsmath,amssymb}
\usepackage{amsthm}
\usepackage{fullpage}
\usepackage{graphicx}
\usepackage{url}
\usepackage{xspace}
\usepackage{listings}
\usepackage{xcolor}
\usepackage{hyperref}
\usepackage{eurosym}
\usepackage{pdfpages}
\usepackage{tikz}
%\usepackage[yyyymmdd,hhmmss]{datetime}
%\usepackage{dblfloatfix}

%\usepackage{float}
%\floatstyle{boxed}
%\restylefloat{figure}

%configuration de listings pour l'affichage du code
\lstset{
    language=ruby,
    basicstyle=\ttfamily\small, %
    identifierstyle=\color{red}, %
    keywordstyle=\color{blue}, %
    stringstyle=\color{black!60}, %
    commentstyle=\it\color{black!50}, %
    columns=flexible, %
    tabsize=3, %
    extendedchars=true, %
    showspaces=false, %
    showstringspaces=false, %
    %numbers=left, %
    %numberstyle=\tiny, %
    breaklines=true, %
    breakautoindent=true, %
    captionpos=b,
    frame=single
}

% On fera des listes a puce et non a tiret.
%\renewcommand{\FrenchLabelItem}{\textbullet}

\newtheorem{theoreme}{Th\'{e}or\`{e}me}
\newtheorem{definition}{D\'{e}finition}
\newtheorem{exercice}{Exercice}

\newcommand{\tab}{\hspace*{\parindent}}

%Definition de quelques commandes utiles en maths :

% R et N
\newcommand{\R}{\mathbb{R}}
\newcommand{\N}{\mathbb{N}}
\newcommand{\C}{\mathbb{C}}
% derive partielle et congruence
\newcommand{\drond}{\partial}
\newcommand{\congru}{\equiv}
% blocs parenthese, valeur absolue, norme, crochets, accolades
\newcommand{\abs}[1]{\left\lvert#1\right\rvert}
\newcommand{\norm}[1]{\left\lVert#1\right\lVert}
\newcommand{\braces}[1]{\left(#1\right)}
\newcommand{\croch}[1]{\left[#1\right]}
\newcommand{\cbraces}[1]{\left\{#1\right\}}
% blocs parties entieres superieur et inferieur
\newcommand{\entsup}[1]{\left\lceil#1\right\rceil}
\newcommand{\entinf}[1]{\left\lfloor#1\right\rfloor}

\newcommand{\HRule}{\rule{\linewidth}{0.5mm}}

\newcommand{\gitInfo}{
    \IfFileExists{.compil/gitfile.tex}
    {
        \input{.compil/gitfile.tex}
        \begin{tabular}{|ll|}
            \hline
            Commit  & \GITAbrHash \\
            Date    & \GITAuthorDate \\
            Author  & \GITAuthorName \\
            \hline
            Build date    & \today \\
            Build time    & \currenttime \\
            \hline
        \end{tabular}
    }{
        \begin{tabular}{|ll|}
            \hline
            No git info \\
            \hline
            Build date    & \today \\
            Build time    & \currenttime \\
            \hline
        \end{tabular}
    }
}

%\setcounter{secnumdepth}{3}
\setlength{\columnsep}{10mm}


\title{Systèmes distribués : Makefile}
\author{Thibaut Coutelou, Benjamin Michel, Guillaume Perrin, Nicolas Vignes}
\date{\today}

\begin{document}
\maketitle
%\tableofcontents

\setlength{\parskip}{2mm}

\section{Introduction}
Ce rapport a pour but de présenter le travail réalisé dans le cadre du projet de systèmes distribués, en présentant notamment, le langage que nous avons utilisé et les algorithmes que nous avons utilisés.

Dans une deuxième partie, nous présenterons les performances de notre application au travers des différents tests que nous avons réalisés. Cette partie se constituera principalement de courbes présentant les résultats.

\section{Partie 1 : Présentation du projet}
\subsection{Langage}
Pour réaliser ce projet, nous avons utilisé le langage \textbf{Go}, développé par Google. Il s'agit d'un langage impératif et concurrent, inspiré du C et visant à de bonnes vitesses d'exécution tout en diminuant le temps du compilation par rapport au C. Go propose la librairie \textbf{RPC} (Remote Protocol Call) qui permet l'appel à des fonctions à distance (d'une instance Go à une autre). Go n'est pas présent nativement sur les machines de l'Ensimag, il faut donc installer le compilateur ainsi que les bibliothèques sur sa machine. Cela se réalise facilement en se rendant sur le site \href{http://golang.org/doc/install#download}{golang.org}.

\subsection{Algorithmes}
Le fonctionnement de notre application de makefile distribués se déroulent en deux parties :
\begin{itemize}
\item Lecture et construction de l'arbre de dépendances à partir du Makefile par le maître.
\item Exécution des différentes tâches par les esclaves.

La première étape, va permettre de construire l'arbre de dépendances à partir du Makefile.
% TODO

\end{itemize}

\subsection{Structure du code}
Le code source est décomposé en plusieurs fichiers .go visant à créer deux exécutables : listener et client.
\begin{itemize}
\item listener.go : écoute des clients et lancement de workers
\item worker.go : exécution des commandes pour générer les cibles
\item client.go : appel à config et au parser pour les options de config et l'arbre des dépandances puis appels aux workers pour créer les différentes cibles
\item config.go : lecture du fichier de configuration pour préciser les hosts
\item parser.go : construction de l'arbre des dépendances d'un Makefile
\end{itemize}

\subsection{Déploiement}
La méthode pour déployer l'application est entièrement disponible dans le fichier \textit{readme.md} disponible à la racine du projet.

\section{Partie 2 : Performances}

\end{document}

% vim: set spell spelllang=fr:
